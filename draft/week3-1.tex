\documentclass{article}
\usepackage{amsmath,amsfonts,amsthm,amssymb}
\usepackage{setspace}
\usepackage{fancyhdr}
\usepackage{lastpage}
\usepackage{extramarks}
\usepackage{extarrows}
\usepackage{chngpage}
\usepackage{soul,color}
\usepackage{mathrsfs}
\usepackage[linesnumbered,boxed]{algorithm2e}  %for using box
\usepackage{graphicx,float,wrapfig}     %for inserting figures etc.
\usepackage{verbatim}   %for using the environment of comment
\usepackage{multicol}    %for using the environment of multicolumn
\usepackage{tikz}           %for convenient drawing
\usetikzlibrary{arrows,decorations.pathmorphing,backgrounds,positioning,fit,petri}
\usepackage{pdfpages}
\usepackage[pdftex,colorlinks=true,linkcolor=black,urlcolor=blue]{hyperref}

%BASIC INFO
\newcommand{\StudentName}{Wu Yijie}
\newcommand{\StudentClass}{JK30}
\newcommand{\StudentNumber}{2013011314}
\newcommand{\Class}{Quantum Information}
\newcommand{\ClassInstructor}{Xiongfeng Ma}


%%%% In case you need to adjust margins:
\topmargin=-0.75in      %
\evensidemargin=0in     %
\oddsidemargin=0in      %
\textwidth=6.5in        %
\textheight=9.5in       %
\headsep=0.10in         %

%%%% Setup the header and footer
\pagestyle{fancy}                                                       %
                     %
\chead{\Title\quad\firstxmark}  %
\rhead{Page\ \thepage\ of\ \protect\pageref{LastPage}}                                                     %
\lfoot{\lastxmark}                                                      %
\cfoot{}                                                                %
\rfoot{}             %
\renewcommand\headrulewidth{0.4pt}                                      %
\renewcommand\footrulewidth{0pt}                                      %

%%%%%%%%%%%%%%%%%%%%%%%%%%%%%%%%%%%%%%%%%%%%%%%%%%%%%%%%%%%%%
% Some tools
\newcommand{\enterProblemHeader}[1]{\nobreak\extramarks{#1}{#1 continued on next page\ldots}\nobreak%
                                    \nobreak\extramarks{#1 (continued)}{#1 continued on next page\ldots}\nobreak}%
\newcommand{\exitProblemHeader}[1]{\nobreak\extramarks{#1 (continued)}{#1 continued on next page\ldots}\nobreak%
                                   \nobreak\extramarks{#1}{}\nobreak}%

\newcommand{\homeworkProblemName}{}%
\newcounter{homeworkProblemCounter}%
\newenvironment{hwPro}[1][Problem \arabic{homeworkProblemCounter}]%
  {\stepcounter{homeworkProblemCounter}%
   \renewcommand{\homeworkProblemName}{#1}%
   \section*{\homeworkProblemName}%
   \enterProblemHeader{\homeworkProblemName}}%
  {\exitProblemHeader{\homeworkProblemName}}%

\newcommand{\homeworkSectionName}{}%
\newlength{\homeworkSectionLabelLength}{}%
\newenvironment{hwSec}[1]%parts of homework problem
  {% We put this space here to make sure we're not connected to the above.

   \renewcommand{\homeworkSectionName}{#1}%
   \settowidth{\homeworkSectionLabelLength}{\homeworkSectionName}%
   \addtolength{\homeworkSectionLabelLength}{0.2in}%
   \changetext{}{-\homeworkSectionLabelLength}{}{}{}%
   \subsection*{\homeworkSectionName}%
   \enterProblemHeader{\homeworkProblemName\ [\homeworkSectionName]}}%
  {\enterProblemHeader{\homeworkProblemName}%

   % We put the blank space above in order to make sure this margin
   % change doesn't happen too soon.
   \changetext{}{+\homeworkSectionLabelLength}{}{}{}}%

\newcommand{\Answer}{\ \\\textbf{Answer:} }   %note first letter 'A' capital!
\newcommand{\Acknowledgement}[1]{\ \\{\bf Acknowledgement:} #1}
\newcommand{\wtM}{\textcolor{white}{M}}  %for indent in before paragraphs
\newcommand{\ud}{\mathrm{d}} %for derivative
\newcommand{\cm}{\mathrm{\,cm}} %centimeter
%%%%%%%%%%%%%%%%%%%%%%%%%%%%%%%%%%%%%%%%%%%%%%%%%%%%%%%%%%%%%

\newcommand{\Title}{Week 2}
\newcommand{\DueDate}{September, 2015}


%% Useful short-cut
\newcommand{\Tr}{\mathrm{Tr}}
\newcommand{\bfone}{\mathbf{1}}
\newcommand{\bra}[1]{\langle #1\vert}
\newcommand{\ket}[1]{\vert #1\vert}
\newcommand{\braket}[2]{\langle #1 \vert #2\rangle}
\newcommand{\ketbra}[2]{\vert #1\rangle \langle #2\vert}


%%%%%%%%%%%%%%%%%%%%%%%%%%%%%%%%%%%%%%%%%%%%%%%%%%%%%%%%%%%%%

\lhead{\StudentName\quad \StudentNumber}
% Make title
\title{\textmd{\bf \Class\\ \Title}\\{\large Instructed by \textit{\ClassInstructor}}\\\normalsize\vspace{0.1in}\small{Due\ on\ \DueDate}}
\date{}
\begin{document}

\begin{spacing}{1.2}
\author{\textbf{\StudentName}\qquad\StudentClass\quad\StudentNumber}
\maketitle \thispagestyle{empty}
\section{Homework 1 due}
Hand in to MMW-S323 mail box 1st floor.
\section{Tomography}
$\rho = \sum_x p(x)\ketbra{x} \otimes f(x)$, e.g. $\frac12 \left[ \ketbra{0} \otimes \rho_0 + \ketbra{1}\otimes \rho_1\right]$.

Measure $x,y,z$ for $\rho = \frac12 (I+\vec{P}\cdot\vec{\sigma})$. $P_i = \Tr(\rho\sigma_i)(i = x,y,z)$.
\begin{gather*}
\rho = \frac12 \left( \Tr(\rho I)I + \sum_{i=x,y,z}\Tr(\sigma_i \rho)\sigma_i\right)
\end{gather*}

\subparagraph{two-qubit system.}
Use $\sigma_i\otimes\sigma$ to measure $\rho_{AB}$. 

For $\frac{1}{\sqrt2}(\ket{00}_{AB}\pm \ket{11}_{AB}), \frac{1}{\sqrt2}(\ket{01}\pm\ket{10})$ (can not be written as tensor product),

$\rho_A = \frac{1}{2}[1 0; 0 1] = \rho_B, \rho_{AB} = \rho_A\otimes\rho_B = \left(\begin{array}{cccc}
&&&\\
&&&\\
&&&\\
&&&
\end{array}\right)_{4\times 4} = \sum_{i,j\in\{I,x,y,z\}}P_{ij}\sigma_i\otimes\sigma_j$.


Beccause $\Tr{\rho_{AB}} = \sum_{i,j}P_{ij}\Tr{\sigma_i\otimes\sigma_j} = P_{II}\Tr(\sigma_I)\Tr(\sigma_I) = 4P_{II}=1$, $P_{II} = \frac14$.

$P_{ij} = \Tr(\sigma_{i'j'}P_{i'j'}(\sigma_{i'}\otimes\sigma_{j'})(\sigma_i\otimes\sigma_j)) = \Tr(\rho\sigma_i\otimes\sigma_j)$.

\subparagraph{n-qubit system.}
$\bigotimes_{v_i=\in\{I,x,y,z\}} \sigma_{v_i}, \rho = \sigma_{v_i}\Tr(\rho\bigotimes_{v_i}\sigma_{v_i})\bigotimes(\sigma_{v_i}).$
$3^n$ measurement should be performed, but the DOF is less, so this tomography is not optimal. 

\subparagraph{Individual measurement:} $4^n$ terms (nothing for $I$).
\subparagraph{Joint measurement (BSM: Bell-state measurement):} $\phi^{\pm} = \ket{00}\pm\ket{11}, \psi^{\pm} = \ket{01}\pm\ket{10}$

Hamard $H = \frac{1}{\sqrt2}[1, 1; 1, -1]$ operation, C-NOT operation (control-NOT, flip target qubit when input control qubit $\ket{1}$, e.g. $\ket{1}\ket{0}\mapsto\ket{1}\ket{1})$) is $4\times 4$ unitary matrix:
\begin{gather*}
\left(\begin{array}{cccc}
1&0&0&0\\
0&1&0&0\\
0&0&0&1\\
0&0&1&0
\end{array}\right).
\end{gather*}
$\ket{00}+\ket{11}\underrightarrow{C-NOT}\ket{+}\ket{0} \underrightarrow{H\times I} \ket{0}\ket{0}$, $\ket{00}-\ket{11}\underrightarrow{C-NOT, H\otimes I} \ket{1}\ket{0}, \ket{01}+\ket{10}\mapsto \ket{0}\ket{1}, \ket{01}-\ket{10}\mapsto \ket{1}\ket{1}, \ket{00}\mapsto\ket{+}\ket{0}$. Then measure in $z$-basis.

\subparagraph{Purification.} Given $\rho_A = \frac{1}{2}(\ketbra{0}+\ketbra{1})$, find $\Psi_{AB}$ s.t. $\Tr_{AB} = \rho_A$. Answer: $\ket{00}+\ket{11}$, not unique, but all purification can be linked by linear operation. e.g.
$ \left(\begin{array}{cc}
\frac49 &\frac{1}{100}\\
\frac{1}{100}&\frac59\\
\end{array}\right)$

For diagonal terms, $\sqrt{\frac49}\ket{0}\ket{0}+\sqrt{\frac59}\ket{1}$. For general matrix, extends $\rho_A$ to pure $\rho_{AB}$ (think about it before checking the book).

\section{Schmidt decomposition.}
For pure $\ket{\psi}_{AB} = \sum\sqrt{p_i}\ket{i}\bra{i'}$, you can always find $\rho_A = \rho_B$ in all basis ($\rho_A = \Tr_B(\psi_{AB})$)

\end{spacing}


\end{document}

