\documentclass{article}
\usepackage{amsmath,amsfonts,amsthm,amssymb}
\usepackage{setspace}
\usepackage{fancyhdr}
\usepackage{lastpage}
\usepackage{extramarks}
\usepackage{extarrows}
\usepackage{chngpage}
\usepackage{soul,color}
\usepackage{mathrsfs}
\usepackage[linesnumbered,boxed]{algorithm2e}  %for using box
\usepackage{graphicx,float,wrapfig}     %for inserting figures etc.
\usepackage{verbatim}   %for using the environment of comment
\usepackage{multicol}    %for using the environment of multicolumn
\usepackage{tikz}           %for convenient drawing
\usetikzlibrary{arrows,decorations.pathmorphing,backgrounds,positioning,fit,petri}
\usepackage{CJK}
\usepackage{pdfpages}
\usepackage[pdftex,colorlinks=true,linkcolor=black,urlcolor=blue]{hyperref}

%BASIC INFO
\newcommand{\StudentName}{Wu Yijie}
\newcommand{\StudentClass}{JK30}
\newcommand{\StudentNumber}{2013011314}
\newcommand{\Class}{Quantum Information}
\newcommand{\ClassInstructor}{Xiongfeng Ma}


%%%% In case you need to adjust margins:
\topmargin=-0.75in      %
\evensidemargin=0in     %
\oddsidemargin=0in      %
\textwidth=6.5in        %
\textheight=9.5in       %
\headsep=0.10in         %

%%%% Setup the header and footer
\pagestyle{fancy}                                                       %
                     %
\chead{\Title\quad\firstxmark}  %
\rhead{Page\ \thepage\ of\ \protect\pageref{LastPage}}                                                     %
\lfoot{\lastxmark}                                                      %
\cfoot{}                                                                %
\rfoot{}             %
\renewcommand\headrulewidth{0.4pt}                                      %
\renewcommand\footrulewidth{0pt}                                      %

%%%%%%%%%%%%%%%%%%%%%%%%%%%%%%%%%%%%%%%%%%%%%%%%%%%%%%%%%%%%%
% Some tools
\newcommand{\enterProblemHeader}[1]{\nobreak\extramarks{#1}{#1 continued on next page\ldots}\nobreak%
                                    \nobreak\extramarks{#1 (continued)}{#1 continued on next page\ldots}\nobreak}%
\newcommand{\exitProblemHeader}[1]{\nobreak\extramarks{#1 (continued)}{#1 continued on next page\ldots}\nobreak%
                                   \nobreak\extramarks{#1}{}\nobreak}%

\newcommand{\homeworkProblemName}{}%
\newcounter{homeworkProblemCounter}%
\newenvironment{hwPro}[1][Problem \arabic{homeworkProblemCounter}]%
  {\stepcounter{homeworkProblemCounter}%
   \renewcommand{\homeworkProblemName}{#1}%
   \section*{\homeworkProblemName}%
   \enterProblemHeader{\homeworkProblemName}}%
  {\exitProblemHeader{\homeworkProblemName}}%

\newcommand{\homeworkSectionName}{}%
\newlength{\homeworkSectionLabelLength}{}%
\newenvironment{hwSec}[1]%parts of homework problem
  {% We put this space here to make sure we're not connected to the above.

   \renewcommand{\homeworkSectionName}{#1}%
   \settowidth{\homeworkSectionLabelLength}{\homeworkSectionName}%
   \addtolength{\homeworkSectionLabelLength}{0.2in}%
   \changetext{}{-\homeworkSectionLabelLength}{}{}{}%
   \subsection*{\homeworkSectionName}%
   \enterProblemHeader{\homeworkProblemName\ [\homeworkSectionName]}}%
  {\enterProblemHeader{\homeworkProblemName}%

   % We put the blank space above in order to make sure this margin
   % change doesn't happen too soon.
   \changetext{}{+\homeworkSectionLabelLength}{}{}{}}%

\newcommand{\Answer}{\ \\\textbf{Answer:} }   %note first letter 'A' capital!
\newcommand{\Acknowledgement}[1]{\ \\{\bf Acknowledgement:} #1}
\newcommand{\wtM}{\textcolor{white}{M}}  %for indent in before paragraphs
\newcommand{\ud}{\mathrm{d}} %for derivative
\newcommand{\cm}{\mathrm{\,cm}} %centimeter
%%%%%%%%%%%%%%%%%%%%%%%%%%%%%%%%%%%%%%%%%%%%%%%%%%%%%%%%%%%%%

\newcommand{\Title}{Week 2}
\newcommand{\DueDate}{September, 2015}

\newcommand{\Tr}{\mathrm{Tr}}

%%%%%%%%%%%%%%%%%%%%%%%%%%%%%%%%%%%%%%%%%%%%%%%%%%%%%%%%%%%%%

\lhead{\StudentName\quad \StudentNumber}
% Make title
\title{\textmd{\bf \Class\\ \Title}\\{\large Instructed by \textit{\ClassInstructor}}\\\normalsize\vspace{0.1in}\small{Due\ on\ \DueDate}}
\date{}
\begin{document}
\begin{CJK}{UTF8}{gkai} %for chinese characters
\begin{spacing}{1.2}
\author{\textbf{\StudentName \quad吴艺杰}\qquad\StudentClass\quad\StudentNumber}
\maketitle \thispagestyle{empty}

\section{Review}
$\vert u\rangle = \cos\frac{\theta}{2}\vert 0\rangle + e^{i\theta} \sin\frac{\theta}{2}\vert 1\rangle$\\
density matrix $\rho = \vert \phi \rangle\langle \phi\vert$\\
$\rho\dagger = \rho$, $\mathrm{Tr}(\rho) = 1$.\\
$\forall \vert \psi\vert\rho\vert\psi\rangle \geq 0,\rho^2 = \rho \Rightarrow \rho(\rho-1)$. Eigenvalue $0$ or $1$.\\
$\langle \psi\vert M\vert \psi\rangle = \mathrm{Tr}(\langle \psi\vert M\vert \psi\rangle) = \mathrm{Tr}( M\vert \psi\rangle \langle \psi\vert) =\mathrm{Tr}(M\rho)$. $M$ is a measurement (projection).\\
Unitary operation. $H = \frac{1}{\sqrt2}\left(\begin{array}{cc}1 & 1\\1 & -1
\end{array}\right)$. $H^\dagger H = \mathbf{1}$.\\
Tensor product $\vert \phi\rangle \vert 0\rangle$.

\section{Two-Qubit System}
\subsection{Notation}
\subparagraph{Tensor Product.}
\begin{gather*}
\left[\begin{array}{cc} a_{11} & a_{12} \\ a_{21}& a_{22}\end{array}\right] \otimes \left[\begin{array}{cc} b_{11} & b_{12}\\ b_{21}& b_{22} \end{array}\right] =
\left[\begin{array}{cc} a_{11} \left[\begin{array}{cc} b_{11} & b_{12}\\ b_{21}& b_{22} \end{array}\right]  & a_{12} \left[\begin{array}{cc} b_{11} & b_{12}\\ b_{21}& b_{22} \end{array}\right]  \\ a_{21}  \left[\begin{array}{cc} b_{11} & b_{12}\\ b_{21}& b_{22} \end{array}\right]  & a_{22}  \left[\begin{array}{cc} b_{11} & b_{12}\\ b_{21}& b_{22} \end{array}\right] \end{array}\right] 
\end{gather*}
Example. $\vert 0\rangle_A \vert 0\rangle_B\qquad \uparrow, \vert 1\rangle_A \vert 1\rangle_B\qquad \downarrow$. $\vert \psi\rangle_{AB} = a\vert 0\rangle_A \vert 0\rangle_B + b\vert 1\rangle_A \vert 1\rangle_B$. Then $\vert \psi\rangle_A = a\vert 0\rangle + b\vert 1\rangle$. Let $a = b = 1/\sqrt2$, then $\vert\psi\rangle_A = \vert +\rangle$.\\
$\vert+\rangle_A\vert +\rangle_B\qquad \rightarrow\rightarrow, \vert -\rangle_A \vert -\rangle_B\qquad \leftarrow \leftarrow$.

$\vert \psi'\rangle_{AB} = (\vert +\rangle\vert +\rangle + \vert -\rangle \vert -\rangle)/\sqrt{2}$. $\vert\psi'\rangle = (\vert +\rangle + \vert -\rangle)/\sqrt2 = \vert 0\rangle$. 
$\vert \psi\rangle_{AB} = \vert \psi'\rangle_{AB}$, but $\vert \psi\rangle_A = \vert\rangle'_A\rangle$.

EPR paradox. If Bob measure by $\sigma_z$ and get $\vert 1\rangle$ then Alice learned that with some probability, so some information transmits instantly (faster than the light speed). Actually, Alice can not rule out any case. no-signaling(stronger, must non-local), non-locality. 

(partial trace) 


\begin{eqnarray*}
\rho_A &=& \mathrm{Tr}_B(\rho_{AB}) = \mathrm{Tr}_B ((\vert 0\rangle_A \vert 0\rangle_B+\vert 1\rangle_A \vert 1\rangle_B)(\langle 0\vert_A \langle 0\vert_B+\langle 1\vert_A \langle 1\vert_B))\\
& = & \vert 0\rangle_A\langle 0\vert_A + \vert 1\rangle_A \langle 1\vert_A = \frac{1}{2} \mathbf{1}.
\end{eqnarray*} .
$\rho_A^2 = \frac{1}{4} \mathbf{1} \neq \rho_A$. Mixed state. ($\rho^2 = \rho$ for pure state.)

\subparagraph{Property}
1. $\rho_A^\dagger = \rho_A$\\
2. $\forall \vert \psi\rangle, \langle \psi\vert \rho_A\vert \psi\rangle\geq 0$.\\
3. $\mathrm{Tr}(\rho_A) = \mathrm{Tr}_A(\Tr_B(\rho_{AB})) = \Tr(\rho_{AB}) = 1$

$\rho_A = a_0\sigma_1 + \ldots$
$\Tr(\rho_A) = 2a_0 = 1\Rightarrow a_0 = \frac{1}{2}$, so $\rho_A = \frac{1}{2}(\sigma_1+\vec{P}\cdot \vec{\sigma})$.

\end{spacing}

\end{CJK}

\end{document}

